
\documentclass[12pt,a4paper]{article}
\usepackage{t1enc}
\usepackage[latin1]{inputenc}
\usepackage[english]{babel}

%%%%%%%%%%%%%%%%%%%%%%%%%%%%%%%%%%%%%%%%%%%%%%%%%%%%%%%%%%%%%%%%%%%%%%%%%%%%%%%%%%%%%%%%%%%%%%%%%%%%%%%%%%%%%%%%%%%%%%%%%%%%%%%%%%%%%%%%%%%%%%%%%%%%%%%%%%%%%%%%%%%%%%%%%%%%%%%%%%%%%%%%%%%%%%%%%%%%%%%%%%%%%%%%%%%%%%%%%%%%%%%%%%%%%%%%%%%%%%%%%%%%%%%%%%%%

\topmargin -1.5cm
 \textheight 26cm
\textwidth 18cm
\oddsidemargin -1cm
\evensidemargin -1cm
%%%%%%%%%%%%%%%%%%%%%%%
\usepackage{pslatex}
\usepackage[active]{srcltx}
\usepackage{pifont}
\usepackage{amsfonts}
\usepackage{babel}
\usepackage{longtable}
%%%%%%%%%%%%%%%%%%%%
\usepackage{pslatex}
\usepackage{graphicx}
\usepackage{amsmath}
\usepackage{amssymb}
\usepackage{graphicx}
\usepackage{fancyhdr}
\usepackage[dvips]{color}
%%%%%%%%%%%%%%%%%%%%
\def\RR{\mathbb R}
\def\NN{\mathbb N}
\renewcommand{\labelenumii}{\roman{enumii})}

\pagestyle{fancy}

\lhead{Modelos y Simulaci\'on}
\chead{Primer cuatrimeste de 2016}
\rhead{U.N.C}
\lfoot{}
\cfoot{}
\rfoot{}

\begin{document}
%\fontfamily{pag}
\fontfamily{ppl}

\begin{center}
\scshape\large\textbf{Pr\'actica 6}
\end{center}
\begin{center}
\small \sc An�lisis estad�stico de datos simulados.
\end{center}

%%%%%%%%%%%%%%%%%%%%%%%%%%%%%%%%%%%%%%%%%%%%%%%%%%%%%%%%%%%%%%%%%%%%%

\noindent \bf{Ejercicio 1.} \rm
Generar~$n$ valores de una variable aleatoria normal
est�ndar de manera tal que se cumplan las condiciones: $n \geq 30$
y $S/\sqrt{n} < 0.1$, siendo~$S$ la desviaci�n est�ndar muestral
de los~$n$ datos generados.
\begin{itemize}
\item[a)] ?`Cu�l es el n�mero esperado de datos que deben
generarse para cumplir las condiciones?
\item[b)] ?`Cu�l es el n�mero de datos generados
efectivamente?
\item[c)] ?`Cu�l es la media muestral de los datos
generados?
\item[d)] ?`Cu�l es la varianza muestral de los datos
generados?
\item[e)] Comente los resultados de los items (c) y (d).
?`Son sorprendentes?
\end{itemize}




\vspace{.3cm}

%%%%%%%%%%%%%%%%%%%%%%%%%%%%%%%%%%%%%%%%%%%%%%%%%%%%%%%%%%%%%%%%%%%%%

\noindent \bf{Ejercicio 2.} \rm
Estimar mediante el m�todo de Monte Carlo la integral
%
\[
\int_0^1 \exp(x^2) \,dx \;.
\]
%
Generar al menos~$100$ valores y detenerse cuando la desviaci�n
est�ndar del estimador sea menor que~$0.01$.

\vspace{.3cm}

%%%%%%%%%%%%%%%%%%%%%%%%%%%%%%%%%%%%%%%%%%%%%%%%%%%%%%%%%%%%%%%%%%%%%

\noindent \bf{Ejercicio 3.} \rm
Para $U_1, U_2, \dots$ variables aleatorias uniformemente
distribu�das en el intervalo~$(0,1)$, se define:
%
$$
N = \mbox{M�nimo} \left\{ n : \sum_{i=1}^n U_i > 1 \right\}
$$
%
Es decir, $N$ es igual a la cantidad de n�meros aleatorios que
deben sumarse para exceder a~$1$.
Como se mostr� en el Problema 3 de la Gu\'{\i}a N$^{\circ}$3,
$E[N] = e$.

\noindent Calcular la varianza del estimador~$\bar{N}$
correspondiente a~$1000$ ejecuciones de la simulaci�n
y dar una estimaci�n de~$e$ mediante un intervalo de
confianza de~$95 \%$.

\vspace{.3cm}

%%%%%%%%%%%%%%%%%%%%%%%%%%%%%%%%%%%%%%%%%%%%%%%%%%%%%%%%%%%%%%%%%%%%%

\noindent \bf{Ejercicio 4.} \rm
Considere una sucesi�n de n�meros aleatorios y sea~$M$ el
primero que es menor que su predecesor. Es decir,
%
\[
M = \{ n : U_1 \leq U_2 \leq \dots \leq U_{n-1} > U_n \}
\]
%
\begin{itemize}
\item[a)] Justificar que~$P(M > n) = 1/n!$, $n \geq 0$.
\item[b)] Utilizar la identidad
%
\[
E[M] = \sum_{n=0}^{\infty} P(M > n)
\]
%
para mostrar que~$E[M] = e$.
\item[c)] Utilizar el resultado del item anterior para
estimar~$e$ mediante~$1000$ ejecuciones de una simulaci�n.
\item[d)] Calcular la varianza del estimador del item~(c)
y dar una estimaci�n de~$e$ mediante un intervalo de confianza
de~$95\%$.
\end{itemize}


\vspace{.3cm}

%%%%%%%%%%%%%%%%%%%%%%%%%%%%%%%%%%%%%%%%%%%%%%%%%%%%%%%%%%%%%%%%%%%%%

\noindent \bf{Ejercicio 5.} \rm
Estimar~$\pi$ sorteando puntos uniformemente distribu�dos
en el cuadrado cuyos v�rtices son: $(1,1)$, $(-1,1)$, $(-1,-1)$,
$(1,-1)$, y contabilizando la fracci�n que cae dentro del
c�rculo inscripto de radio~$1$.

\noindent Obtener un intervalo de ancho menor que~$0.1$, el cual
contenga a~$\pi$ con el~$95 \%$ de confianza.
?`Cu�ntas ejecuciones son necesarias?

\vspace{.3cm}

%%%%%%%%%%%%%%%%%%%%%%%%%%%%%%%%%%%%%%%%%%%%%%%%%%%%%%%%%%%%%%%%%%%%%

\noindent \bf{Ejercicio 6.} \rm
Sean~$X_1,\dots, X_n$ variables aleatorias independientes
e id�nticamente distribu�das con media~$\mu$ desconocida.
Para~$a$ y~$b$ constantes dadas, $a<b$, nos interesa estimar
%
$$p= \displaystyle P \left( a < \sum_{i=1}^n X_i/n - \mu < b \right).$$
%
\begin{itemize}
\item[a)] Explicar como utilizar el m�todo ``bootstrap''
para estimar~$p$.
\item[b)] Estimar~$p$ asumiendo que para~$n=10$, los valores
de las variables~$X_i$ resultan 56, 101, 78, 67, 93, 87, 64, 72, 80
y 69. Sean~$a=-5$ y $b=5$.
\end{itemize}

\vspace{.3cm}

%%%%%%%%%%%%%%%%%%%%%%%%%%%%%%%%%%%%%%%%%%%%%%%%%%%%%%%%%%%%%%%%%%%%%

\noindent \bf{Ejercicio 7.} \rm
Sean~$X_1,\dots, X_n$ variables aleatorias independientes
e id�nticamente distribu�das con varianza~$\sigma^2$
desconocida.
Se planea estimar~$\sigma^2$ mediante la varianza muestral
%
$$S^2 = \displaystyle \sum_{i=1}^n (X_i-\overline{X})^2 / (n-1)$$.
%

\vspace{-0.5cm}
\noindent Si~$n=2$, $X_1=1$ y~$X_2=3$, ?`cu�l es la estimaci�n
``bootstrap'' de~$\mbox{Var}(S^2)$?

\vspace{.3cm}

%%%%%%%%%%%%%%%%%%%%%%%%%%%%%%%%%%%%%%%%%%%%%%%%%%%%%%%%%%%%%%%%%%%%%

\noindent \bf{Ejercicio 8.} \rm
Considerar un sistema con un �nico servidor en el cual los
clientes potenciales llegan de acuerdo con un proceso de Poisson de
raz�n~$4.0$. Un cliente potencial entrar� al sistema s�lo si
hay tres o menos clientes en el sistema al momento de su llegada.
El tiempo de servicio de cada cliente est� distribu�do seg�n
una exponencial de par�metro~$4.2$.
Despues del instante~$T=8$ no entran m�s clientes al sistema (los
tiempos est�n dados en horas). Realizar un estudio de simulaci�n
para estimar el tiempo promedio que un cliente pasa en el sistema.
Aplicar el m�todo ``bootstrap'' para estudiar el error cuadr�tico
medio de su estimador.

\end{document}
